\documentclass[book=true]{jlreq}

\usepackage{makeidx}
\usepackage[linkcolor=true]{hyperref}

\title{The Type Theory of Intheo}
\author{Hexirp}

\begin{document}

\frontmatter

\maketitle

Intheo は、純粋関数型プログラミング言語である。そして、カリー・ハワード対応を利用して定理を証明することも可能である。
となれば、どのような型理論に基づいているのかという疑問は当然であろう。この文書は、その疑問に答えるだろう。

Intheo がベースとする型理論は、 Coq が基礎とする帰納的構造の可述的計算~\cite{cic} と、
方形的型理論~\cite{cutt} と、数量的型理論~\cite{qtt} の三つを一まとめにしたものである。

\tableofcontents

\mainmatter

\part{参考文献}

\begin{thebibliography}
  \bibitem{cic}
    Inria, CNRS, and contributors.
    \href{https://coq.github.io/doc/v8.13/refman/language/cic.html}{\textit{Typing rules --- Coq 8.13.2 documentation}}.
  \bibitem{cutt}
    Cyril Cohen, Thierry Coquand, Simon Huber, and Anders Mörtberg.
    \textit{Cubical Type Theory: a constructive interpretation of the univalence axiom}.
  \bibitem{qtt}
    Robert Atkey.
    \href{https://bentnib.org/quantitative-type-theory.html}{\textit{The Syntax and Semantics of Quantitative Type Theory}}.
\end{thebibliography}

\backmatter

\makeindex

\end{document}
